\documentclass[11pt]{report}
\usepackage[spanish]{babel}
\usepackage[latin1]{inputenc}
\usepackage{amsmath,amsthm,amsfonts,amssymb}
\usepackage{Sweave}
\usepackage{graphicx}
\usepackage{hyperref}
\usepackage{anysize} 
\marginsize{1.78cm}{1.65cm}{1.78cm}{1.78cm} 

\title{\Huge Universidad Nacional de Loja \\ 
?rea de la Energa las Industrias y los Recursos Naturales no Renovables \\
Ingeniera en Sistemas \\}

\author{\includegraphics[width=7cm, height=7cm]{unl.jpg}\\\\ Victor Francisco Jumbo Sinchire \\\\ ECINF7228 \\\\ \texttt{vfjumbos@unl.edu.ec}}

\begin{document}
\input{P3_1104074479-concordance}
\maketitle
\begin{center}\textbf{CUESTIONARIO DE PREGUNTAS}\end{center}
\textbf{Pregunta 1}\\
El CO2 trama de datos tiene 84 filas y 5 columnas de los datos de un experimento en la tolerancia al fr?o de las especies de gram?neas Echinochloa crus-galli.
\\
\begin{center}\textbf{CO2 DATASET}\end{center}
\textbf{Descripci?n}
El CO2 trama de datos tiene 84 filas y 5 columnas de los datos de un experimento en la tolerancia al fr?o de las especies de gram?neas Echinochloa crus-galli.
\\
\textbf{Uso:}

\\
\textbf{Formato}
Un objeto de la clase c("nfnGroupedData", "nfGroupedData", "groupedData", "data.frame") que contiene las siguientes columnas:
\begin{itemize}
\item Planta: Un factor ordenada con niveles Qn1 Qn2 Qn3 <... Mc1 dando un identificador ?nico para cada planta.
\item Tipo: Un factor con los niveles de Quebec Mississippi que da el origen de la planta.
\item Tratamiento: Un factor con los niveles nonchilled refrigerada.
\item Conc: Un vector num?rico de las concentraciones ambientales de di?xido de carbono (mL L).
\item Captacin: Un vector numrico de las tasas de absorcin de di?xido de carbono (umol m  2 seg).
\end{itemize}
\textbf{Detalles}

La absorci?n de CO2 de seis plantas de Quebec y seis plantas de Mississippi se midi? en varios niveles de concentraci?n de CO2 ambiente.  La mitad de las plantas de cada tipo se enfriaron durante la noche antes se llev? a cabo el experimento. Este conjunto de datos fue originalmente parte del paquete nlme y que tiene m?todos (incluyendo por [ as.data.frame plot y print para sus clases de datos agrupados.
\\
\textbf{Fuente}
\begin{itemize}
\item Potvin, C., Lechowicz, M. J. and Tardif, S. (1990) "The statistical analysis of ecophysiological response curves obtained from experiments involving repeated measures", Ecology, 71, 1389-1400. 
\item Pinheiro, J. C. and Bates, D. M. (2000) Mixed-effects Models in S and S-PLUS, Springer.
\end{itemize}
\textbf{Determinar las estad?sticas del Type Quebec y del Treatment chilled.}

